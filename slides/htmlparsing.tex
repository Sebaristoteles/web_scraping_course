\begin{frame}{HTML parsing}
\begin{itemize}
	\item After obtaining the HTML source code, how to obtain the information required?
	\item If the HTML code is well-structured and its tags have (more or less) unique names, we can navigate the HTML elements to get the information we want.
	\item The {\tt beautifulsoup4} package converts the HTML code into a Python object that can be navigated using properties and functions.
\end{itemize}
\end{frame}

\begin{frame}{Some HTML terms}
\begin{itemize}
	\item Consider {\tt <a href="http://www.bccp-berlin.de" target="\_blank">BCCP</a>}
	\item HTML Elements
	\begin{itemize}
		\item The entire thing is an HTML element. Specifically, it is a link leading to the BCCP website and displayed as "BCCP".
		\item HTML elements usually consist of a start tag and an end tag.
	\end{itemize}
	\item HTML Tags
	\begin{itemize}
		\item The start tag of the element above is {\tt <a>} and the end tag is the corresponding {\tt </a>}
		\item Start tag can and sometimes must contain attributes.
	\end{itemize}
	\item HTML Attributes
	\begin{itemize}
		\item The {\tt <a>} tag contains the attribute {\tt href} and {\tt target}. {\tt href} specifies the destination to which the link should lead and {\tt target="\_blank"} specifies that the link should be opened in a new window.
		\item For web scraping purposes, the attributes {\tt class} and {\tt id} are usually useful as these are often used to identify certain (groups of) elements.
	\end{itemize}
\end{itemize}
\end{frame}

\begin{frame}[fragile]{Basic HTML documents structure}
\begin{itemize}
	\item HTML documents have a tree-like/nested structure
	\item Elements can contain various levels of sub-elements that in the end contain some content
	\item A very simple HTML document could look like this:
\end{itemize}
\begin{verbatim}
<html>
<body>
  Hi all! <br>
  Do you know 
  <a href="http://www.bccp-berlin.de" target="_blank">BCCP</a>?
</body>
</html>
\end{verbatim}
\end{frame}

\begin{frame}{Tree structure}
\begin{tikzpicture}
\draw (0,0) node {\verb{<html>}};
\end{tikzpicture}
\end{frame}

\begin{frame}{Example for today}
\begin{itemize}
	\item Let's scrape the details of all upcoming BCCP events: \url{http://www.bccp-berlin.de/events/all-events/}
	\item Steps:
		\begin{enumerate}
			\item Analyze HTML structure
			\item Save information on events available on the front page
			\item Loop through individual event pages to get details
			\item Combine to DataFrame
		\end{enumerate}
\end{itemize}
\end{frame}