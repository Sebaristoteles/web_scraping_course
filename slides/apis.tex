
\begin{frame}
    \frametitle{Application Programming Interface}

    Why APIs?

    \begin{itemize}
        \item Data owners want know who is using their services.
        \item Data owners want to limit requests.
        \item Data owners want to supply data in their preferred format.
    \end{itemize}
\end{frame}

\subsubsection*{Twitter API}
\label{sec:twitter-api}

\begin{frame}
    \frametitle{Twitter API}
    \begin{itemize}
        \item ``Conduct historical research and search from Twitter's
        massive archive of publicly-available Tweets posted since
        March 2006?''
        \item ``Listen in real-time for Tweets of interest?''
    \end{itemize}
    Source: \href{https://developer.twitter.com/en/docs/basics/getting-started.html}{https://developer.twitter.com/en/docs/basics/getting-started.html}
\end{frame}

\begin{frame}
    \frametitle{Twitter API --- Limits}

    All request windows are 15 minutes in length.

    \begin{table}[]
        \footnotesize
        \begin{tabular}{llll}
          \textbf{Endpoint}           & \textbf{Resource family} & \textbf{Requests / window (user auth)} \\
          GET followers/list          & followers                & 15                                     \\
          GET lists/members           & lists                    & 900                                    \\
          GET lists/statuses          & lists                    & 900                                    \\
          GET search/tweets           & search                   & 180                                    \\
          GET statuses/lookup         & statuses                 & 900                                    \\
          GET statuses/retweeters/ids & statuses                 & 75                                     \\
          GET statuses/user\_timeline & statuses                 & 900                                    \\
          GET users/lookup            & users                    & 900                                    \\
        \end{tabular}
    \end{table}

    Next to request windows other restrictions may
    apply (e.g. statuses/user\_timeline has an additional restriction of
    the last 3200 tweets).

    Source: \href{https://developer.twitter.com/en/docs/basics/rate-limits}{https://developer.twitter.com/en/docs/basics/rate-limits}
\end{frame}

\begin{frame}[fragile]
    \frametitle{Tweepy package}

    We use the Tweepy package to access twitter's RESTful API.

\begin{minted}{python}
user = api.get_user('twitter')

# tweepy models contain the data plus and some methods.
print(user.screen_name)
print(user.followers_count)
for friend in user.friends():
    print(friend.screen_name)
\end{minted}

\end{frame}

\begin{frame}[fragile]
    \frametitle{Twitter API --- JSON Example}

    Packages usually also allow to access the JSON directly, which
    often contains more information than provided by the API.

    \begin{minted}{python}
import tweepy
from twitter_auth import auth

def get_tweets(api, screen_name):
    tweets_json = [status._json for status in tweepy.Cursor(
        api.user_timeline,
        screen_name=screen_name,
        tweet_mode='extended'
    ).items(2)]
    return tweets_json

api = tweepy.API(auth)
tweets = get_tweets(api, '@guardian')
    \end{minted}
\end{frame}

\begin{frame}[fragile]
    \frametitle{Twitter API --- JSON Example}
\begin{minted}[fontsize=\tiny]{python}
{'contributors': None,
 'coordinates': None,
 'created_at': 'Tue Nov 20 17:56:53 +0000 2018',
 'display_text_range': [0, 97],
 'entities': {'hashtags': [],
              'symbols': [],
              'urls': [{'display_url': 'trib.al/hDWAWvZ',
                        'expanded_url': 'https://trib.al/hDWAWvZ',
                        'indices': [74, 97],
                        'url': 'https://t.co/GpWbVaZV3F'}],
              'user_mentions': []},
 'favorite_count': 17,
 'favorited': False,
 'full_text': 'I was arrested at a climate change protest – it was worth it | '
              'Gavin Turk https://t.co/GpWbVaZV3F',
 'geo': None,
 'id': 1064940660942352385,
 'id_str': '1064940660942352385',
 'in_reply_to_screen_name': None,
 'in_reply_to_status_id': None,
 'in_reply_to_status_id_str': None,
 'in_reply_to_user_id': None,
 'in_reply_to_user_id_str': None,
 'is_quote_status': False,
 'lang': 'en',
 'place': None,
 'possibly_sensitive': False,
 'retweet_count': 6,
 'retweeted': False,
 'source': '<a href="http://www.socialflow.com" rel="nofollow">SocialFlow</a>',
 'truncated': False,
 'user': {'contributors_enabled': False,
          'created_at': 'Thu Nov 05 23:49:19 +0000 2009',
          'default_profile': False,
          'default_profile_image': False,
          'description': 'The need for independent journalism has never been '
                         'greater. Become a Guardian supporter: '
                         'https://t.co/gWyuUVlObq',
          'entities': {'description': {'urls': [{'display_url': 'support.theguardian.com',
                                                 'expanded_url': 'https://support.theguardian.com',
                                                 'indices': [89, 112],
                                                 'url': 'https://t.co/gWyuUVlObq'}]},
                       'url': {'urls': [{'display_url': 'theguardian.com',
                                         'expanded_url': 'https://www.theguardian.com',
                                         'indices': [0, 23],
                                         'url': 'https://t.co/c53pnmnuIT'}]}},
          'favourites_count': 147,
          'follow_request_sent': False,
          'followers_count': 7409250,
          'following': False,
          'friends_count': 1084,
          'geo_enabled': False,
          'has_extended_profile': False,
          'id': 87818409,
          'id_str': '87818409',
          'is_translation_enabled': True,
          'is_translator': False,
          'lang': 'en',
          'listed_count': 58265,
          'location': 'London',
          'name': 'The Guardian',
          'notifications': False,
          'profile_background_color': 'FFFFFF',
          'profile_background_image_url': 'http://abs.twimg.com/images/themes/theme1/bg.png',
          'profile_background_image_url_https': 'https://abs.twimg.com/images/themes/theme1/bg.png',
          'profile_background_tile': False,
          'profile_banner_url': 'https://pbs.twimg.com/profile_banners/87818409/1542013526',
          'profile_image_url': 'http://pbs.twimg.com/profile_images/1061907978633297921/aPuDuMXq_normal.jpg',
          'profile_image_url_https': 'https://pbs.twimg.com/profile_images/1061907978633297921/aPuDuMXq_normal.jpg',
          'profile_link_color': '005789',
          'profile_sidebar_border_color': 'FFFFFF',
          'profile_sidebar_fill_color': 'CAE3F3',
          'profile_text_color': '333333',
          'profile_use_background_image': False,
          'protected': False,
          'screen_name': 'guardian',
          'statuses_count': 484822,
          'time_zone': None,
          'translator_type': 'regular',
          'url': 'https://t.co/c53pnmnuIT',
          'utc_offset': None,
          'verified': True}}
\end{minted}

\end{frame}


\begin{frame}
    \frametitle{World Bank API}
    \begin{itemize}
        \item World Bank APIs provide access to:
        \begin{itemize}
            \item Indicators API
            \item Data Catalog API
            \item Projects API
            \item Finances API
            \item Climate Data API
        \end{itemize}
        \item Access data without authentication.
        \item \href{https://datahelpdesk.worldbank.org/knowledgebase/articles/889386-developer-information-overview}{World bank API documentation}
        \item \href{https://github.com/mwouts/world_bank_data}{world\_bank\_data package documentation}
    \end{itemize}
\end{frame}

\begin{frame}[fragile]
    \frametitle{world\_bank\_data --- Example}
\begin{minted}{python}
import world_bank_data as wb

# Get estimates for the world population:
wb.get_series('SP.POP.TOTL', date='2017')

# Get timeseries of "Agricultural machinery, tractors" in Albania
wb.get_series('AG.AGR.TRAC.NO', country='ALB')
\end{minted}
\end{frame}

\begin{frame}
    \frametitle{There might be APIs without a working package}
    \begin{itemize}
        \item Check more general packages. For example, \href{https://pandas-datareader.readthedocs.io/en/latest/readers/}{https://pandas-datareader.readthedocs.io/en/latest/readers/}
        \item Write your own API wrappers.
    \end{itemize}
\end{frame}

\begin{frame}
    \frametitle{RESTful API}
    \begin{itemize}
        \item Most APIs are RESTful APIs (Representational State Transfer)
        \item RESTful APIs use HTTP methods:
        \begin{itemize}
            \item GET --- fetch item
            \item POST --- create item
            \item DELETE --- delete item
            \item PUT --- modify an existing item
        \end{itemize}
    \end{itemize}
\end{frame}

\begin{frame}[fragile]
    \frametitle{RESTful API --- Example}

    For web scraping we only need GET.

\begin{minted}{python}
import requests

url = ('http://ec.europa.eu/eurostat/wdds/rest/data/v2.1/json/en/'
       'nama_10_gdp?geo=EU28&precision=1&na_item=B1GQ&unit=CP_MEUR&'
       'time=2010&time=2011')

resp = requests.get(url)
resp_json = resp.json()

resp_json['value']
resp_json['dimension']['time']['category']['index']
\end{minted}
\end{frame}