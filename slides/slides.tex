
\documentclass[english]{beamer}
\usepackage{appendixnumberbeamer}
\usetheme[
progressbar=foot,
sectionpage=progressbar,
subsectionpage=progressbar
]{metropolis}
\useoutertheme{metropolis}
\useinnertheme{metropolis}
\usefonttheme{metropolis}
\usecolortheme{seahorse}
\setbeamercolor{background canvas}{bg=white}
\setbeamercolor{progress bar}{fg=cyan}
\setbeamerfont{caption}{size=\tiny}
\makeatletter
\setlength{\metropolis@progressinheadfoot@linewidth}{1pt}
\makeatother

% Other packages
\usepackage[export]{adjustbox}% http://ctan.org/pkg/adjustbox
\usepackage{mathptmx}
\usepackage[T1]{fontenc}
\usepackage[utf8]{inputenc}
\usepackage{amsmath}
\usepackage{amssymb}
\usepackage{graphicx}
% \usepackage{eurosym}
\usepackage{hyperref}
% \usepackage{media9}
% \usepackage{menukeys}
\usepackage{lmodern,textcomp}
\usepackage{color}
\usepackage{tabulary}
\usepackage{booktabs}

\usepackage{babel}
% \usepackage{pgf}
\begin{document}
\title[Web scraping]{BCCP web scraping course}
\titlegraphic{}

\author[]{}

\date[]{}

\begin{frame}
    \titlepage
\end{frame}

\section{Day 1}
\label{sec:day1}

\begin{frame}
    \frametitle{Table of Contents}
    \tableofcontents[currentsection]
\end{frame}

\subsection{Very short intro to Python}
\label{sec:intropython}

\begin{frame}
    \frametitle{Your experiences}
    \begin{itemize}
        \item Which tools or programming languages do you use when
        working with data?
        \item Have you used Python before?
    \end{itemize}
\end{frame}

\begin{frame}
    \frametitle{Why Python for Webscraping?}
    \begin{itemize}
        \item<1:> Common web data structures are similar to data
        structures in Python.
        \item<2:> Many Python packages for webscraping and APIs can be
        found.
    \end{itemize}
\end{frame}

\begin{frame}[fragile]
    \frametitle{Python interpreter --- interactive mode}
    \begin{itemize}
        \item input prompt \mintinline{python}{>>>}
        \item comments \mintinline{python}{#}
        \item operators \mintinline{python}{+}, \mintinline{python}{-}, \mintinline{python}{*} and \mintinline{python}{/}
    \end{itemize}
\begin{minted}{python}
>>> 2 + 2
4
>>> 8 / 5  # division always returns a floating point number
1.6
>>> 5 ** 2  # 5 squared
25
\end{minted}
\end{frame}

\begin{frame}[fragile]
    \frametitle{Python data structures}
    \begin{itemize}
        \item Lists (\mintinline{python}{value})\\
\begin{minted}{python}
squares = [1, 4, 9, 16, 25]
\end{minted}
        \item Dictionaries (\mintinline{python}{key: value})\\
\begin{minted}{python}
followers = {'kevin': 15, 'julian': 9}
\end{minted}
    \end{itemize}
\end{frame}

\begin{frame}[fragile]
    \frametitle{Python data structures --- Lists}
\begin{minted}{python}
>>> squares = [1, 4, 9, 16, 25]
>>> squares[0]  # indexing returns the item
1
>>> squares[-1]
25
>>> squares[-3:]  # slicing returns a new list
[9, 16, 25]
>>> squares.append((len(squares) + 1) ** 2)  # using append() method
>>> squares
[1, 4, 9, 16, 25, 36]
\end{minted}
\end{frame}

\begin{frame}[fragile]
    \frametitle{Python data structures --- Dictionaries}
    \begin{itemize}
        \item Dictionaries (\mintinline{python}{key: value})\\
        \item Unlike lists, dictionaries are indexed by keys not by positions.
\begin{minted}{python}
>>> followers = {'kevin': 15, 'julian': 9}
>>> followers['kevin']
15
>>> followers['kevin'] = 16
\end{minted}
    \end{itemize}
\end{frame}

\begin{frame}[fragile]
    \frametitle{Common web data structures}
        For example, JSON nearly identical to combination of Python's dictionaries and lists.
\begin{minted}{json}
[
  {
    "id": "1290837412912998347",
    "followers": "15",
    "name": "Kevin"
  },
  {
    "id": "12908374129734908043",
    "followers": "9",
    "name": "Julian"
  }
]
\end{minted}
\end{frame}

\begin{frame}[fragile]
    \frametitle{Functions}
    \begin{itemize}
        \item  \mintinline{python}{def} defines a function.
        \item Followed by function name with parenthesized sequence of parameters.
        \item Body of function must be indented.
    \end{itemize}
\begin{minted}{python}
>>> def list_append(a, mylist=[]):
...     """Example documentation string:
...     Append value to list. mylist defaults to empty list."""
...     mylist.append(a)
...     return mylist

>>> list_append(97, [99, 98])
[99, 98, 97]

>>> list_append(1)
[1]
\end{minted}
\end{frame}

\begin{frame}[fragile]
    \frametitle{Modules}
    \begin{itemize}
        \item Definitions (functions and variables) can be saved
        in \textbf{modules}. Our example function can be saved under \mintinline{python}{list_operations.py}.
        \item Such modules can be imported into the interpreter,
        scripts, or other modules.
\begin{minted}{python}
import list_operations
list_operations.list_append(1)

from list_operations import list_append
list_append(1)

import list_operations as lo
lo.list_append(1)
\end{minted}
        \item Bad practice: \mintinline{python}{from sound.effects import *}
    \end{itemize}
\end{frame}

\begin{frame}[fragile,fragile]
    \frametitle{Packages}
    \begin{itemize}
        \item Packages structure modules namespace by using dotted
        module names. \mintinline{python}{A.B} designates a submodule
        named \mintinline{python}{B} in a package named
        \mintinline{python}{A}.
\begin{minted}{python}
import matplotlib.pyplot as plt
\end{minted}
        \item Packages can be installed with pip.
\begin{minted}{bash}
pip install matplotlib
\end{minted}
        \item A common convention is to have a list of packages in requirements.txt:
\begin{minted}{bash}
pip install -r requirements.txt
\end{minted}
    \end{itemize}
\end{frame}

\begin{frame}[fragile]
    \frametitle{Files}
    \begin{itemize}
        \item Reading and writing files
\begin{minted}{python}
f = open('workfile', 'r') # read-only
f = open('workfile', 'w') # write-only
f = open('workfile', 'a') # appending

f = open('workfile') # read-only, as mode defaults to 'r'
\end{minted}
        \item Closing files
\begin{minted}{python}
f.close() # Manually close a file

with open('workfile') as f: # with closes file "automatically"
     read_data = f.read()
\end{minted}
        \item Several packages offer file operations
        \mintinline{python}{pandas.read_csv()}
    \end{itemize}
\end{frame}

% \begin{frame}
%     \frametitle{Further reading}
%     \begin{itemize}
%         \item
%     \end{itemize}
% \end{frame}


\subsection{Intro to Webscraping}
\label{sec:introwebscraping}

\subsection{APIs}
\label{sec:apis}

\begin{frame}
    \frametitle{Application Programming Interface}

    Why APIs?

    \begin{itemize}
        \item Data owners want know who is using their services.
        \item Data owners want to limit requests.
        \item Data owners want to supply data in their preferred format.
    \end{itemize}
\end{frame}

\subsubsection*{Twitter API}
\label{sec:twitter-api}

\begin{frame}
    \frametitle{Twitter API}
    \begin{itemize}
        \item ``Conduct historical research and search from Twitter's
        massive archive of publicly-available Tweets posted since
        March 2006?''
        \item ``Listen in real-time for Tweets of interest?''
    \end{itemize}
    Source: \href{https://developer.twitter.com/en/docs/basics/getting-started.html}{https://developer.twitter.com/en/docs/basics/getting-started.html}
\end{frame}

\begin{frame}
    \frametitle{Twitter API --- Limits}

    All request windows are 15 minutes in length.

    \begin{table}[]
        \footnotesize
        \begin{tabular}{llll}
          \textbf{Endpoint}           & \textbf{Resource family} & \textbf{Requests / window (user auth)} \\
          GET followers/list          & followers                & 15                                     \\
          GET lists/members           & lists                    & 900                                    \\
          GET lists/statuses          & lists                    & 900                                    \\
          GET search/tweets           & search                   & 180                                    \\
          GET statuses/lookup         & statuses                 & 900                                    \\
          GET statuses/retweeters/ids & statuses                 & 75                                     \\
          GET statuses/user\_timeline & statuses                 & 900                                    \\
          GET users/lookup            & users                    & 900                                    \\
        \end{tabular}
    \end{table}

    Next to request windows other restrictions may
    apply (e.g. statuses/user\_timeline has an additional restriction of
    the last 3200 tweets).

    Source: \href{https://developer.twitter.com/en/docs/basics/rate-limits}{https://developer.twitter.com/en/docs/basics/rate-limits}
\end{frame}

\begin{frame}[fragile]
    \frametitle{Tweepy package}

    We use the Tweepy package to access twitter's RESTful API.

\begin{minted}{python}
user = api.get_user('twitter')

# tweepy models contain the data plus and some methods.
print(user.screen_name)
print(user.followers_count)
for friend in user.friends():
    print(friend.screen_name)
\end{minted}

\end{frame}

\begin{frame}[fragile]
    \frametitle{Twitter API --- JSON Example}

    Packages usually also allow to access the JSON directly, which
    often contains more information than provided by the API.

    \begin{minted}{python}
import tweepy
from twitter_auth import auth

def get_tweets(api, screen_name):
    tweets_json = [status._json for status in tweepy.Cursor(
        api.user_timeline,
        screen_name=screen_name,
        tweet_mode='extended'
    ).items(2)]
    return tweets_json

api = tweepy.API(auth)
tweets = get_tweets(api, '@guardian')
    \end{minted}
\end{frame}

\begin{frame}[fragile]
    \frametitle{Twitter API --- JSON Example}
\begin{minted}[fontsize=\tiny]{python}
{'contributors': None,
 'coordinates': None,
 'created_at': 'Tue Nov 20 17:56:53 +0000 2018',
 'display_text_range': [0, 97],
 'entities': {'hashtags': [],
              'symbols': [],
              'urls': [{'display_url': 'trib.al/hDWAWvZ',
                        'expanded_url': 'https://trib.al/hDWAWvZ',
                        'indices': [74, 97],
                        'url': 'https://t.co/GpWbVaZV3F'}],
              'user_mentions': []},
 'favorite_count': 17,
 'favorited': False,
 'full_text': 'I was arrested at a climate change protest – it was worth it | '
              'Gavin Turk https://t.co/GpWbVaZV3F',
 'geo': None,
 'id': 1064940660942352385,
 'id_str': '1064940660942352385',
 'in_reply_to_screen_name': None,
 'in_reply_to_status_id': None,
 'in_reply_to_status_id_str': None,
 'in_reply_to_user_id': None,
 'in_reply_to_user_id_str': None,
 'is_quote_status': False,
 'lang': 'en',
 'place': None,
 'possibly_sensitive': False,
 'retweet_count': 6,
 'retweeted': False,
 'source': '<a href="http://www.socialflow.com" rel="nofollow">SocialFlow</a>',
 'truncated': False,
 'user': {'contributors_enabled': False,
          'created_at': 'Thu Nov 05 23:49:19 +0000 2009',
          'default_profile': False,
          'default_profile_image': False,
          'description': 'The need for independent journalism has never been '
                         'greater. Become a Guardian supporter: '
                         'https://t.co/gWyuUVlObq',
          'entities': {'description': {'urls': [{'display_url': 'support.theguardian.com',
                                                 'expanded_url': 'https://support.theguardian.com',
                                                 'indices': [89, 112],
                                                 'url': 'https://t.co/gWyuUVlObq'}]},
                       'url': {'urls': [{'display_url': 'theguardian.com',
                                         'expanded_url': 'https://www.theguardian.com',
                                         'indices': [0, 23],
                                         'url': 'https://t.co/c53pnmnuIT'}]}},
          'favourites_count': 147,
          'follow_request_sent': False,
          'followers_count': 7409250,
          'following': False,
          'friends_count': 1084,
          'geo_enabled': False,
          'has_extended_profile': False,
          'id': 87818409,
          'id_str': '87818409',
          'is_translation_enabled': True,
          'is_translator': False,
          'lang': 'en',
          'listed_count': 58265,
          'location': 'London',
          'name': 'The Guardian',
          'notifications': False,
          'profile_background_color': 'FFFFFF',
          'profile_background_image_url': 'http://abs.twimg.com/images/themes/theme1/bg.png',
          'profile_background_image_url_https': 'https://abs.twimg.com/images/themes/theme1/bg.png',
          'profile_background_tile': False,
          'profile_banner_url': 'https://pbs.twimg.com/profile_banners/87818409/1542013526',
          'profile_image_url': 'http://pbs.twimg.com/profile_images/1061907978633297921/aPuDuMXq_normal.jpg',
          'profile_image_url_https': 'https://pbs.twimg.com/profile_images/1061907978633297921/aPuDuMXq_normal.jpg',
          'profile_link_color': '005789',
          'profile_sidebar_border_color': 'FFFFFF',
          'profile_sidebar_fill_color': 'CAE3F3',
          'profile_text_color': '333333',
          'profile_use_background_image': False,
          'protected': False,
          'screen_name': 'guardian',
          'statuses_count': 484822,
          'time_zone': None,
          'translator_type': 'regular',
          'url': 'https://t.co/c53pnmnuIT',
          'utc_offset': None,
          'verified': True}}
\end{minted}

\end{frame}


\begin{frame}
    \frametitle{World Bank API}
    \begin{itemize}
        \item World Bank APIs provide access to:
        \begin{itemize}
            \item Indicators API
            \item Data Catalog API
            \item Projects API
            \item Finances API
            \item Climate Data API
        \end{itemize}
        \item Access data without authentication.
        \item \href{https://datahelpdesk.worldbank.org/knowledgebase/articles/889386-developer-information-overview}{Worldbank API documentation}
        \item \href{https://github.com/mwouts/world_bank_data}{world\_bank\_data package documentation}
    \end{itemize}
\end{frame}

\begin{frame}[fragile]
    \frametitle{world\_bank\_data --- Example}
\begin{minted}{python}
import world_bank_data as wb

# Get estimates for the world population:
wb.get_series('SP.POP.TOTL', date='2017')

# Get timeseries of "Agricultural machinery, tractors" in Albania
wb.get_series('AG.AGR.TRAC.NO', country='ALB')
\end{minted}
\end{frame}

\begin{frame}
    \frametitle{There might be APIs without a working package}
    \begin{itemize}
        \item Check more general packages. For example, \href{https://pandas-datareader.readthedocs.io/en/latest/readers/}{https://pandas-datareader.readthedocs.io/en/latest/readers/}
        \item Write your own API wrappers.
    \end{itemize}
\end{frame}

\begin{frame}
    \frametitle{RESTful API}
    \begin{itemize}
        \item Most APIs are RESTful APIs (Representational State Transfer)
        \item RESTful APIs use HTTP methods:
        \begin{itemize}
            \item GET --- fetch item
            \item POST --- create item
            \item DELETE --- delete item
            \item PUT --- modify an existing item
        \end{itemize}
    \end{itemize}
\end{frame}

\begin{frame}[fragile]
    \frametitle{RESTful API --- Example}

    For web scraping we only need GET.

\begin{minted}{python}
import requests

url = ('http://ec.europa.eu/eurostat/wdds/rest/data/v2.1/json/en/'
       'nama_10_gdp?geo=EU28&precision=1&na_item=B1GQ&unit=CP_MEUR&'
       'time=2010&time=2011')

resp = requests.get(url)
resp_json = resp.json()

resp_json['value']
resp_json['dimension']['time']['category']['index']
\end{minted}
\end{frame}

\section{Day 2}

\begin{frame}
    \frametitle{Table of Contents}
    \tableofcontents[currentsection]
\end{frame}

\subsection{HTML parsing}
\begin{frame}{HTML parsing}
\begin{itemize}
	\item After obtaining the HTML source code, how to obtain the information required?
	\item If the HTML code is well-structured and its tags have (more or less) unique names, we can navigate the HTML elements to get the information we want.
	\item The {\tt beautifulsoup4} package converts the HTML code into a Python object that can be navigated using properties and functions.
\end{itemize}
\end{frame}

\begin{frame}{Some HTML terms}
\begin{itemize}
	\item Consider {\tt <a href="http://www.bccp-berlin.de" target="\_blank">BCCP</a>}
	\item HTML Elements
	\begin{itemize}
		\item The entire thing is an HTML element. Specifically, it is a link leading to the BCCP website and displayed as "BCCP".
		\item HTML elements usually consist of a start tag and an end tag.
	\end{itemize}
	\item HTML Tags
	\begin{itemize}
		\item The start tag of the element above is {\tt <a>} and the end tag is the corresponding {\tt </a>}
		\item Start tag can and sometimes must contain attributes.
	\end{itemize}
	\item HTML Attributes
	\begin{itemize}
		\item The {\tt <a>} tag contains the attribute {\tt href} and {\tt target}. {\tt href} specifies the destination to which the link should lead and {\tt target="\_blank"} specifies that the link should be opened in a new window.
		\item For web scraping purposes, the attributes {\tt class} and {\tt id} are usually useful as these are often used to identify certain (groups of) elements.
	\end{itemize}
\end{itemize}
\end{frame}

\begin{frame}[fragile]{Basic HTML documents structure}
\begin{itemize}
	\item HTML documents have a tree-like/nested structure
	\item Elements can contain various levels of sub-elements that in the end contain some content
	\item A very simple HTML document could look like this:
\end{itemize}
\begin{verbatim}
<html>
<body>
  Hi all! <br>
  Do you know 
  <a href="http://www.bccp-berlin.de" target="_blank">BCCP</a>?
</body>
</html>
\end{verbatim}
\end{frame}

\begin{frame}{Example for today}
\begin{itemize}
	\item Let's scrape the details of all upcoming BCCP events: \url{http://www.bccp-berlin.de/events/all-events/}
	\item Steps:
		\begin{enumerate}
			\item Analyze HTML structure
			\item Save information on events available on the front page
			\item Loop through individual event pages to get details
			\item Combine to DataFrame
		\end{enumerate}
\end{itemize}
\end{frame}

\subsection{Text pattern matching}

\section{Day 3}

\begin{frame}
    \frametitle{Table of Contents}
    \tableofcontents[currentsection]
\end{frame}

\subsection{Browser automation}

\subsection{Own script}

\end{document}
%%% Local Variables:
%%% mode: xelatex
%%% TeX-master: t
%%% End:
