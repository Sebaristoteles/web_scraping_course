
\documentclass[english,aspectratio=169]{beamer}
\usepackage{appendixnumberbeamer}
\usetheme[
progressbar=foot,
sectionpage=progressbar,
subsectionpage=progressbar
]{metropolis}
\useoutertheme{metropolis}
\useinnertheme{metropolis}
\usefonttheme{metropolis}
\usecolortheme{seahorse}
\setbeamercolor{background canvas}{bg=white}
\setbeamercolor{progress bar}{fg=cyan}
\setbeamerfont{caption}{size=\tiny}
\makeatletter
\setlength{\metropolis@progressinheadfoot@linewidth}{1pt}
\makeatother

% Other packages
\usepackage[export]{adjustbox}% http://ctan.org/pkg/adjustbox
\usepackage{mathptmx}
\usepackage[T1]{fontenc}
\usepackage[utf8]{inputenc}
\usepackage{amsmath}
\usepackage{amssymb}
\usepackage{graphicx}
% \usepackage{eurosym}
\usepackage{hyperref}
% \usepackage{media9}
% \usepackage{menukeys}
\usepackage{lmodern,textcomp}
\usepackage{color}
\usepackage{tabulary}
\usepackage{booktabs}

\usepackage{babel}
% \usepackage{pgf}
\begin{document}
\title[Web scraping]{BCCP Web Scraping Course}
\titlegraphic{}

\author[]{}

\date[]{}

\begin{frame}
    \titlepage
\end{frame}

\section{Day 1}
\label{sec:day1}

% General intro here?

\begin{frame}
    \frametitle{Table of Contents}
    \tableofcontents[currentsection]
\end{frame}

\subsection{very short intro to Python}
\label{sec:intropython}

\begin{frame}
    \frametitle{Very short intro to Python}
    \begin{itemize}
        \item
    \end{itemize}
\end{frame}

\subsection{Intro to Webscraping}
\label{sec:introwebscraping}

\begin{frame}{Introduction to web scraping}
\begin{itemize}
	\item Basic idea: Turn information on website to structured data
	% Screenshot of webpage and corresponding data
	\item Typical workflow:
	\begin{enumerate}
		\item Look at website to decide best approach
		\begin{itemize}
			\item Is an Application Programming Interface (API) available?
			\item Do the HTML elements have fixed names?
			\item Does the page load statically or dynamically?
		\end{itemize}
		\item Download information from URL
		\item Turn information into structured data and save
	\end{enumerate}
\end{itemize}
\end{frame}

\begin{frame}{Some concepts}
\begin{itemize}
	\item APIs
	\item HTML parsing vs text matching
	\item Static vs dynamic websites
\end{itemize}
\end{frame}

\begin{frame}{APIs}
\begin{itemize}
	\item If available, a convenient way to get pre-structured data (usually JSON or XML).
	\item Example: OpenStreetMap (OSM) (\url{https://www.openstreetmap.org})
	\begin{itemize}
		\item When searching manually, results can be shown as XML. Automating the search on OpenStreetMap and clicking on the relevant links would therefore be a way to save this data.
		\item However, OSM offers several APIs that simplify this task. One API is the Nominatim API (\url{https://nominatim.openstreetmap.org}).
	\end{itemize}
\end{itemize}
\end{frame}

\begin{frame}{API example: Nominatim API for OSM}
\begin{itemize}
	\item See \url{https://nominatim.org/release-docs/develop/api/Search/} for documentation on search syntax
	\item Search for 'diw berlin' and return as JSON: \url{https://nominatim.openstreetmap.org/search?q=diw+berlin&format=json}
\end{itemize}

\end{frame}

\begin{frame}{HTML parsing}
% Example of website screenshot and HTML code
\begin{itemize}
	\item Use structure of HTML code to find needed information.
	\item Works best if the code is well-structured and element names are fixed.
\end{itemize}
\end{frame}

\begin{frame}{HTML parsing example: eBay search results}
\begin{itemize}
	\item Look at results for 'star wars blu ray' on eBay: \url{https://www.ebay.de/sch/i.html?_nkw=star+wars+blu+ray}
	\item Most browsers have a feature to look at source code (e.g. in Chrome, you can right click on any website element and click on 'Inspect').
	\item On eBay, the HTML tags containing certain content always have the same name, this simplifies HTML parsing.
	\item Foe example, the tag \mintinline{html}{<div id="ResultSetItems">} contains all results. Inside this tag, the individual listings are saved in tags called \mintinline{html}{<li class="sresult">}. In Chrome, you can also look for elements using the XPATH syntax (e.g. for the individual listings: \textit{//li[contains(@class,'sresult')]}). More information on XPATH here: \url{https://www.w3schools.com/xml/xpath_syntax.asp}
\end{itemize}
\end{frame}

\begin{frame}{Text pattern matching}
% Example of website screenshot and HTML code with random names
\begin{itemize}
	\item If the HTML code is not well-structured or names change, text pattern matching is an alternative.
	\item Idea: Take text from (parts of) a page and find needed information by matching a regular expression
\end{itemize}
\end{frame}

\begin{frame}{Example of website without clear HTML tag names: Airbnb}
\begin{itemize}
	\item Search for homes in Berlin-Mitte: \url{https://www.airbnb.de/s/Berlin-Mitte--Berlin/homes?query=Berlin-Mitte\%2C\%20Berlin}
	\item Say you wanted to get the number of results for this search. The element does not have a clear name. Using HTML parsing is still possible but is prone to errors. Instead, one could match on a regular expression.
\end{itemize}

\end{frame}

\begin{frame}{Static vs dynamic websites}
% Example of website screenshot and HTML code with random names
\begin{itemize}
	\item On static websites, the entire content is loaded immediately. E.g. eBay: \url{https://www.ebay.de/sch/i.html?_nkw=star+wars+blu+ray}
	\item On dynamic websites, content may not load instantaneously or only after user action, making them usually more complicated to scrape. E.g. Airbnb: \url{https://www.airbnb.de/s/Berlin-Mitte--Berlin/homes?query=Berlin-Mitte\%2C\%20Berlin} (Try disabling JavaScript in your browser and reloading the page).
	% Show examples of static and dynamic websites and what happens if loading with requests
	\item Getting the complete source code from a dynamic website can be done with browser automation. The idea is to open a website in an actual browser (and interacting with it if necessary) and save the source code of the content from there.
\end{itemize}
\end{frame}

\begin{frame}{Important Python packages}
\begin{itemize}
	\item {\tt requests}: To load URL and recover source code (for static web pages)
	\item {\tt beautifulsoup4}: To turn HTML code to navigable Python object
	\item {\tt selenium}: For browser automation
	\item {\tt pandas}: To create DataFrames
\end{itemize}

\end{frame}

\subsection{APIs}
\label{sec:apis}

\begin{frame}
    \frametitle{Application Programming Interface}
    \begin{itemize}
        \item
    \end{itemize}
\end{frame}

\begin{frame}
    \frametitle{Twitter API}
    \begin{itemize}
        \item "Conduct historical research and search from Twitter's massive
        archive of publicly-available Tweets posted since March 2006?"
        \item "Listen in real-time for Tweets of interest?"
    \end{itemize}
\end{frame}



\section{Day 2}

\begin{frame}
    \frametitle{Table of Contents}
    \tableofcontents[currentsection]
\end{frame}

\subsection{HTML parsing}

\begin{frame}{HTML parsing}
\begin{itemize}
	\item After obtaining the HTML source code, how to obtain the information required?
	\item If the HTML code is well-structured and its tags have (more or less) unique names, we can navigate the HTML elements to get the information we want.
	\item The {\tt beautifulsoup4} package converts the HTML code into a Python object that can be navigated using properties and functions.
\end{itemize}
\end{frame}

\begin{frame}{Some HTML terms}
\begin{itemize}
	\item Consider {\tt <a href="http://www.bccp-berlin.de" target="\_blank">BCCP</a>}
	\item HTML Elements
	\begin{itemize}
		\item The entire thing is an HTML element. Specifically, it is a link leading to the BCCP website and displayed as "BCCP".
		\item HTML elements usually consist of a start tag and an end tag.
	\end{itemize}
	\item HTML Tags
	\begin{itemize}
		\item The start tag of the element above is {\tt <a>} and the end tag is the corresponding {\tt </a>}
		\item Start tag can and sometimes must contain attributes.
	\end{itemize}
	\item HTML Attributes
	\begin{itemize}
		\item The {\tt <a>} tag contains the attribute {\tt href} and {\tt target}. {\tt href} specifies the destination to which the link should lead and {\tt target="\_blank"} specifies that the link should be opened in a new window.
		\item For web scraping purposes, the attributes {\tt class} and {\tt id} are usually useful as these are often used to identify certain (groups of) elements.
	\end{itemize}
\end{itemize}
\end{frame}

\begin{frame}[fragile]{Basic HTML documents structure}
\begin{itemize}
	\item HTML documents have a tree-like/nested structure
	\item Elements can contain various levels of sub-elements that in the end contain some content
	\item A very simple HTML document could look like this:
\end{itemize}
\begin{verbatim}
<html>
<body>
  Hi all! <br>
  Do you know 
  <a href="http://www.bccp-berlin.de" target="_blank">BCCP</a>?
</body>
</html>
\end{verbatim}
\end{frame}

\begin{frame}{Example for today}
\begin{itemize}
	\item Let's scrape the details of all upcoming BCCP events: \url{http://www.bccp-berlin.de/events/all-events/}
	\item Steps:
		\begin{enumerate}
			\item Analyze HTML structure
			\item Save information on events available on the front page
			\item Loop through individual event pages to get details
			\item Combine to DataFrame
		\end{enumerate}
\end{itemize}
\end{frame}

\subsection{Text pattern matching}

\section{Day 3}

\begin{frame}
    \frametitle{Table of Contents}
    \tableofcontents[currentsection]
\end{frame}

\subsection{Browser automation}

\begin{frame}[fragile]{Why browser automation?}
\begin{itemize}
	\item If the content of a page is loaded dynamically (e.g. with JavaScript), using \verb!requests! could yield an ``empty'' source code.
	\item Browser automation is then a way to load the page in an actual browser and let the JavaScript load as if you actually visited the page.
	\item Because this uses an actual browser and a browser driver, this approach is less stable and crashes can occur. Further, loading a page in a browser usually takes more time then loading it in \verb!requests!.
\end{itemize}
\end{frame}

\begin{frame}[fragile]{Example for today}
\begin{itemize}
	\item Let us scrape all future events from the BERA website: \url{https://www.berlin-econ.de/events}.
	\item In order to load all events, we need to click on the bottom buttons to navigate through the results pages.
	\item However, these buttons do not link to a new URL but load content using JavaScript:
\end{itemize}
\begin{verbatim}
<a href="javascript:;" class="item" data-request-success="scroll(0,0)" 
data-request="onEventSearch" 
data-request-update="'@events-list': '#event-results'" 
data-request-data="page:2">Next →</a>
\end{verbatim}
\end{frame}

\begin{frame}[fragile]{Some technical notes}
\begin{itemize}
	\item We will use the \verb!selenium! package
	\begin{itemize}
		\item It allows you to control a browser from a Python script
		\item The documentation can be found here: \url{https://selenium-python.readthedocs.io/}
	\end{itemize}
	\item Besides \verb!selenium!, you need to have an actual browser installed that you are going to use and a compatible browser driver that \verb!selenium! can use to control the browser
	\begin{itemize}
		\item We will use Google's Chrome browser (\url{https://www.google.com/chrome/}) and the corresponding ChromeDriver (\url{http://chromedriver.chromium.org/}). Some parts of the code might have a different syntax for different browsers.
		\item \verb!selenium!'s documentation includes links to drivers for four popular browsers: \url{https://selenium-python.readthedocs.io/installation.html#drivers}
		\item The documentation for the various browser driver types in \verb!selenium! can be found here: \url{https://seleniumhq.github.io/selenium/docs/api/py/api.html}
		\item Make sure that the driver version fits your installed browser version
	\end{itemize}
\end{itemize}
\end{frame}

\begin{frame}[fragile]{First, analyze the HTML code of \url{https://www.berlin-econ.de/events}}
\begin{itemize}
	\item Events are saved in a \verb!<div class='event-results'>! element
	\item Inside this, events for different days are separated by a \verb!<div class='event-date-separator'>! element
	\item The actual events are then saved in a \verb!<div class='ui segments'>! elements, more specifically, in \verb!<div class='ui segment'>! elements
	\item The buttons to navigate to the next results pages are saved in the last element in \verb!<div class='event-results'>! (\verb!<div class='ui pagination menu'>!)
	\item Need a mix of navigating and searching the HTML document
\end{itemize}
\end{frame}

\begin{frame}[fragile]{Approach}
\begin{enumerate}
	\item Load events page in browser
	\item Loop through elements in \verb!<div class='event-results'>!
	\begin{enumerate}
		\item If it is a date, save the date
		\item If it is an event, save the event details
		\item If it is the buttons, click the button for the next page, if available.
	    \item Repeat until no other next page available
	\end{enumerate}
	\item Turn to DataFrame and save
\end{enumerate}
\begin{itemize}
	\item See \verb!automation.ipynb!
\end{itemize}
\end{frame}

\begin{frame}[fragile]{Interacting with the webpage}
\begin{itemize}
	\item In order to be able to click the button, we need to scroll it into view first
	\item For this, we need to tell \verb!selenium! where the wanted element is and have it scroll there
	\item This can be done e.g. using XPATH syntax
	\item Typical steps are therefore:
	\begin{enumerate}
		\item Find the element in the source code (e.g. \verb!element = driver.find_element_by_xpath(xpath)!, other alternatives here: \url{https://selenium-python.readthedocs.io/locating-elements.html})
		\item Scroll it into view and click, e.g. \verb!ActionChains(driver).move_to_element(element).click(element).perform()!
	\end{enumerate}
	\item See \url{https://seleniumhq.github.io/selenium/docs/api/py/webdriver/selenium.webdriver.common.action_chains.html} for documentation on \verb!ActionChains! and things you can do with it 
\end{itemize}
\end{frame}

\begin{frame}{Waits}
\begin{itemize}
	\item It can occur that the page is not finished loading when the script continues and converts the source code
	\item To prevent this, Waits can be used
	\item There are two main types of Waits:
	\begin{itemize}
		\item Explicit Waits: Explicitly waits until a condition is fulfilled or a maximum time is reached
		\item Implicit Waits: Usually set once and is a maximum waiting time whenever some element is looked for
	\end{itemize}
	\item More details here: \url{https://selenium-python.readthedocs.io/waits.html}
\end{itemize}
\end{frame}

\begin{frame}{Explicit Waits with Expected Conditions}
\begin{itemize}
	\item What often comes in handy in browser automation are Explicit Waits with Expected Conditions
	\item Here, you can let the script pause until e.g. some element is visible on the web page
	\item Selenium features some methods that should be enough for most use cases: See Section 7.39 at \url{https://selenium-python.readthedocs.io/api.html}
\end{itemize}
\end{frame}

\begin{frame}[fragile]{Finding the right button}
\begin{itemize}
	\item The page buttons are saved as children of the \verb!<div class='ui pagination menu'>! tag.
	\item Their tags are of the form \verb!<a class="item">!.
	\item Unfortunately, the ``Next'' button does not have a unique id/name.
	\item However, using \verb!find_all()!, we can find the list of \verb!<a class="item">! items, look at the last one, and determine if it is a ``Next'' button or not
\end{itemize}
\end{frame}

\subsection{Own script}

\end{document}
%%% Local Variables:
%%% mode: xelatex
%%% TeX-master: t
%%% End:
