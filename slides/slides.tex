
\documentclass[english]{beamer}
\usepackage{appendixnumberbeamer}
\usetheme[
progressbar=foot,
sectionpage=progressbar,
subsectionpage=progressbar
]{metropolis}
\useoutertheme{metropolis}
\useinnertheme{metropolis}
\usefonttheme{metropolis}
\usecolortheme{seahorse}
\setbeamercolor{background canvas}{bg=white}
\setbeamercolor{progress bar}{fg=cyan}
\setbeamerfont{caption}{size=\tiny}
\makeatletter
\setlength{\metropolis@progressinheadfoot@linewidth}{1pt}
\makeatother

% Other packages
\usepackage[export]{adjustbox}% http://ctan.org/pkg/adjustbox
\usepackage{mathptmx}
\usepackage[T1]{fontenc}
\usepackage[utf8]{inputenc}
\usepackage{amsmath}
\usepackage{amssymb}
\usepackage{graphicx}
% \usepackage{eurosym}
\usepackage{hyperref}
% \usepackage{media9}
% \usepackage{menukeys}
\usepackage{lmodern,textcomp}
\usepackage{color}
\usepackage{tabulary}
\usepackage{booktabs}

\usepackage{babel}
% \usepackage{pgf}
\begin{document}
\title[Web scraping]{BCCP web scraping course}
\titlegraphic{}

\author[]{}

\date[]{}

\begin{frame}
    \titlepage
\end{frame}

\section{Day 1}
\label{sec:day1}

\begin{frame}
    \frametitle{Table of Contents}
    \tableofcontents[currentsection]
\end{frame}

\subsection{very short intro to Python}
\label{sec:intropython}

\begin{frame}
    \frametitle{Very short intro to Python}
    \begin{itemize}
        \item
    \end{itemize}
\end{frame}

\subsection{Intro to Webscraping}
\label{sec:introwebscraping}

\subsection{APIs}
\label{sec:apis}

\begin{frame}
    \frametitle{Application Programming Interface}
    \begin{itemize}
        \item
    \end{itemize}
\end{frame}

\begin{frame}
    \frametitle{Twitter API}
    \begin{itemize}
        \item "Conduct historical research and search from Twitter's massive
        archive of publicly-available Tweets posted since March 2006?"
        \item "Listen in real-time for Tweets of interest?"
    \end{itemize}
\end{frame}



\section{Day 2}

\begin{frame}
    \frametitle{Table of Contents}
    \tableofcontents[currentsection]
\end{frame}

\subsection{HTML parsing}
\subsection{Text pattern matching}

\section{Day 3}

\begin{frame}
    \frametitle{Table of Contents}
    \tableofcontents[currentsection]
\end{frame}

\subsection{Browser automation}
\subsection{Own script}

\end{document}
%%% Local Variables:
%%% mode: xelatex
%%% TeX-master: t
%%% End:
