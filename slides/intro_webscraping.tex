\begin{frame}{Introduction to Webscraping}
\begin{itemize}
	\item Basic idea: Turn information on website to structured data
	% Screenshot of webpage and corresponding data
	\item Typical workflow:
	\begin{enumerate}
		\item Look at website to decide best approach
		\begin{itemize}
			\item Is an Application Programming Interface (API) available?
			\item Do the HTML elements have fixed names?
			\item Does the page load statically or dynamically?
		\end{itemize}
		\item Download information from URL
		\item Turn information into structured data and save
	\end{enumerate}
\end{itemize}
\end{frame}

\begin{frame}{Some concepts}
\begin{itemize}
	\item APIs
	\item HTML parsing vs text matching
	\item Static vs dynamic websites
\end{itemize}
\end{frame}

\begin{frame}{APIs}
\begin{itemize}
	\item If available, a convenient way to get pre-structured data (usually JSON or XML).
	\item Example: OpenStreetMap (OSM) (\url{https://www.openstreetmap.org})
	\begin{itemize}
		\item When searching manually, results can be shown as XML. Automating the search on OpenStreetMap and clicking on the relevant links would therefore be a way to save this data.
		\item However, OSM offers several APIs that simplify this task. One API is the Nominatim API (\url{https://nominatim.openstreetmap.org}).
	\end{itemize}
\end{itemize}
\end{frame}

\begin{frame}{API example: Nominatim API for OSM}
\begin{itemize}
	\item See \url{https://nominatim.org/release-docs/develop/api/Search/} for documentation on search syntax
	\item Search for 'diw berlin' and return as JSON: \url{https://nominatim.openstreetmap.org/search?q=diw+berlin&format=json}
	\item The JSON format has a similar structure as dictionaries in Python and can easily be transformed to DataFrames.
\end{itemize}

\end{frame}

\begin{frame}{HTML parsing}
% Example of website screenshot and HTML code
\begin{itemize}
	\item Use structure of HTML code to find needed information.
	\item Works best if the code is well-structured and element names are fixed.
\end{itemize}
\end{frame}

\begin{frame}{HTML parsing example: eBay search results}
\begin{itemize}
	\item Look at results for 'star wars blu ray' on eBay: \url{https://www.ebay.de/sch/i.html?_nkw=star+wars+blu+ray}
	\item Most browsers have a feature to look at source code (e.g. in Chrome, you can right click on any website element and click on 'Inspect').
	\item On eBay, the HTML tags containing certain content always have the same name, this simplifies HTML parsing.
	\item Foe example, the tag \textit{<div id="ResultSetItems">} contains all results. Inside this tag, the individual listings are saved in tags called \textit{<li class="sresult">}. In Chrome, you can also look for elements using the XPATH syntax (e.g. for the individual listings: \textit{//li[contains(@class,'sresult')]}). More information on XPATH here: \url{https://www.w3schools.com/xml/xpath_syntax.asp}
\end{itemize}
\end{frame}

\begin{frame}{Text pattern matching}
% Example of website screenshot and HTML code with random names
\begin{itemize}
	\item If the HTML code is not well-structured or names change, text pattern matching is an alternative.
	\item Idea: Take text from (parts of) a page and find needed information by matching a regular expression
\end{itemize}
\end{frame}

\begin{frame}{Example of website without clear HTML tag names: Airbnb}
\begin{itemize}
	\item Search for homes in Berlin-Mitte: \url{https://www.airbnb.de/s/Berlin-Mitte--Berlin/homes?query=Berlin-Mitte\%2C\%20Berlin}
	\item Say you wanted to get the number of results for this search. The element does not have a clear name. Using HTML parsing is still possible but is prone to errors. Instead, one could match on a regular expression.
\end{itemize}

\end{frame}

\begin{frame}{Static vs dynamic websites}
% Example of website screenshot and HTML code with random names
\begin{itemize}
	\item On static websites, the entire content is loaded immediately. E.g. eBay: \url{https://www.ebay.de/sch/i.html?_nkw=star+wars+blu+ray}
	\item On dynamic websites, content may not load instantaneously or only after user action, making them usually more complicated to scrape. E.g. Airbnb: \url{https://www.airbnb.de/s/Berlin-Mitte--Berlin/homes?query=Berlin-Mitte\%2C\%20Berlin} (Try disabling JavaScript in your browser and reloading the page).
	% Show examples of static and dynamic websites and what happens if loading with requests
	\item Getting the complete source code from a dynamic website can be done with browser automation. The idea is to open a website in an actual browser (and interacting with it if necessary) and save the source code of the content from there.
\end{itemize}
\end{frame}

\begin{frame}{Important Python packages}
\begin{itemize}
	\item {\tt requests}: To load URL and recover source code (for static web pages)
	\item {\tt beautifulsoup4}: To turn HTML code to navigable Python object
	\item {\tt selenium}: For browser automation
	\item {\tt pandas}: To create DataFrames
\end{itemize}

\end{frame}